\documentclass[12pt,twocolumn]{article}
\usepackage{amsmath, graphicx, float, hyperref, booktabs}
\usepackage{asa} % ASA style file (ensure asa.sty is in the directory)
\usepackage{geometry}
\geometry{margin=1in}

\title{A Regression Analysis of Immutable X Orders}

\begin{document}

\maketitle

\section{Introduction}
\label{sec:introduction}
% Placeholder: Describe the context, goals, and importance of your analysis.

I haven't really thought about blockchain and NFT's much since their quick rise and fall in 2021. However, I became interested in the space 
again, when I learned that there were still projects pulling in millions of dollars in revenue...which seemed incomprehensible to me, 
given the overall bad press that the space has received in recent years and what, from my vantage point, seemed like a real failure to find 
a use case that resonated with the general public. 

One particular project of interest to me was a collection on Immutable X, called Gods Unchained Cards. It's a digitial collectible card game
card game that is close in similar in gameplay to Magic the Gathering or Hearthstone. The cards themselves are NFTs and are stored on the Ethereum blockchain. 
This means that players also own their cards, just like physical cards in traditional games. Players also have the option speculate on the value 
of their cards by buying, selling or trading their cards via the Immutable X marketplace. 



\section{Data Description}
\label{sec:data}
% Placeholder: Describe the source, structure, and cleaning of the data.

\section{Methods}
\label{sec:methods}
% Placeholder: Explain the statistical methods (GAM, bootstrapping) and their formulation.

\section{Results}
\label{sec:results}
% Placeholder: Present key results including tables and figures.

\section{Discussion}
\label{sec:discussion}
% Placeholder: Interpret the results and explain their significance.

\section{Conclusion + Areas of Future Analysis}
\label{sec:conclusion}
% Placeholder: Summarize the main findings and highlight future directions.

\end{document}
